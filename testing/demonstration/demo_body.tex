
A point with a mark and its bounding box

\Huge

\begin{center}
   \input{Fig_QRXCooUmnlhkvh.pstricks}
\end{center}
\normalsize
\input{Fig_QRXCooUmnlhkvh.comment}


This is a circle with a tangent vector and a mark :

\begin{center}
    \input{Fig_TRJEooPRoLnEiG.pstricks}
\end{center}
\input{Fig_TRJEooPRoLnEiG.comment}

Et une autre : 

\begin{center}
   \input{Fig_UREIooqNGBXtHg.pstricks}
\end{center}
\input{Fig_UREIooqNGBXtHg.comment}

For the numbering on the axes with dilatation :


\begin{center}
   \input{Fig_QIXEooejrojKjo.pstricks}
\end{center}
\input{Fig_QIXEooejrojKjo.comment}

\clearpage

\begin{center}
\input{Fig_DEIToomZFknFmn.pstricks}
\end{center}
\input{Fig_DEIToomZFknFmn.comment}


La mantisse :

The result is on figure \ref{LabelFigUARHooLMWqvyaI}. % From file UARHooLMWqvyaI
\newcommand{\CaptionFigUARHooLMWqvyaI}{<+Type your caption here+>}
\input{Fig_UARHooLMWqvyaI.pstricks}
\input{Fig_UARHooLMWqvyaI.comment}

\clearpage

\begin{center}
\input{Fig_HELQooLGapRQrr.pstricks}
\end{center}
\input{Fig_HELQooLGapRQrr.comment}

Autre :

\begin{center}
   \input{Fig_YJEDoojDtSeKHQ.pstricks}
\end{center}
   \input{Fig_YJEDoojDtSeKHQ.comment}


\clearpage

\begin{center}
   \input{Fig_AxesSecond.pstricks}
\end{center}
   \input{Fig_AxesSecond.comment}

   \begin{center}
       \input{Fig_exCircle.pstricks}
   \end{center}
       \input{Fig_exCircle.comment}


\begin{center}
   \input{Fig_exCircleThree.pstricks}
\end{center}
   \input{Fig_exCircleThree.comment}


   \begin{center}
\input{Fig_exCircleTwo.pstricks}
   \end{center}
\input{Fig_exCircleTwo.comment}


One red point :

\begin{center}
   \input{Fig_OnePoint.pstricks}
\end{center}

\begin{center}
\input{Fig_GestionRepere.pstricks}
\end{center}
\input{Fig_GestionRepere.comment}

\begin{center}
\input{Fig_GridOne.pstricks}
\end{center}
\input{Fig_GridOne.comment}

\begin{center}
\input{Fig_GridThree.pstricks}
\end{center}
\input{Fig_GridThree.comment}

\begin{center}
\input{Fig_GridTwo.pstricks}
\end{center}
\input{Fig_GridTwo.comment}


Line

\begin{center}
\input{Fig_Lines.pstricks}
\end{center}
\input{Fig_Lines.comment}

\clearpage

Mark

\begin{center}
\input{Fig_MarkOnPoint.pstricks}
\end{center}
\input{Fig_MarkOnPoint.comment}

\begin{center}
\input{Fig_ParametricOne.pstricks}
\end{center}
\input{Fig_ParametricOne.comment}

\begin{center}
\input{Fig_ParametricTwo.pstricks}
\end{center}
\input{Fig_ParametricTwo.comment}

\begin{center}
\input{Fig_Sequence.pstricks}
\end{center}
\input{Fig_Sequence.comment}

\begin{center}
\input{Fig_VectorOne.pstricks}
\end{center}
\input{Fig_VectorOne.comment}



\begin{center}
   \input{Fig_QRJOooKZPUoLlF.pstricks}
\end{center}
   \input{Fig_QRJOooKZPUoLlF.comment}

Sudoku grid.

\begin{center}
   \input{Fig_RVKFooDxrqYXAX.pstricks}
\end{center}
   \input{Fig_RVKFooDxrqYXAX.comment}

   \clearpage

Multiple subfigures :

There are 4 subfigures at figure \ref{LabelFigPFCUoorQhitKoJ}. % From file PFCUoorQhitKoJ
\newcommand{\CaptionFigPFCUoorQhitKoJ}{4 subfigures still to be descripted.}
\input{Fig_PFCUoorQhitKoJ.pstricks}
\input{Fig_PFCUoorQhitKoJ.comment}

\clearpage

About adding plotpoints :


The result is on figure \ref{LabelFigEXIIooJzzoJeai}. % From file EXIIooJzzoJeai
\newcommand{\CaptionFigEXIIooJzzoJeai}{<+Type your caption here+>}
\input{Fig_EXIIooJzzoJeai.pstricks}
See also the subfigure \ref{LabelFigEXIIooJzzoJeaissLabelSubFigEXIIooJzzoJeai0}
See also the subfigure \ref{LabelFigEXIIooJzzoJeaissLabelSubFigEXIIooJzzoJeai1}
See also the subfigure \ref{LabelFigEXIIooJzzoJeaissLabelSubFigEXIIooJzzoJeai2}


