
\info{GGHOookMhIxqIK}
\begin{center}
   \input{Fig_GGHOookMhIxqIK.pstricks}
\end{center}
   \input{Fig_GGHOookMhIxqIK.comment}
