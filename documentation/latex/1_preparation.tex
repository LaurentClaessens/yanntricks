%+++++++++++++++++++++++++++++++++++++++++++++++++++++++++++++++++++++++++++++++++++++++++++++++++++++++++++++++++++++++++++ 
\section{Preparation}
%+++++++++++++++++++++++++++++++++++++++++++++++++++++++++++++++++++++++++++++++++++++++++++++++++++++++++++++++++++++++++++

%--------------------------------------------------------------------------------------------------------------------------- 
\subsection{Dependencies and installation}
%---------------------------------------------------------------------------------------------------------------------------

\begin{enumerate}
    \item
        You need a working \href{ http://sagemath.org }{ sage } installation.
    \item
        Download \phystricks from \href{ https://github.com/LaurentClaessens/phystricks }{ github } and make it available from Sage (\info{from phystricks import *} has to work).
    \item
        I don't even speak about having a working \LaTeX\ installation with Tikz installed.
\end{enumerate}

%--------------------------------------------------------------------------------------------------------------------------- 
\subsection{In your \LaTeX\ file}
%---------------------------------------------------------------------------------------------------------------------------

The preamble of your \LaTeX\ file has to contain

\begin{verbatim}
    \usepackage{calc}   
    \usepackage{tikz}
    \usetikzlibrary{patterns}
    \usetikzlibrary{calc}
    \newcounter{defHatch}
    \newcounter{defPattern}
    \setcounter{defHatch}{0}
    \setcounter{defPattern}{0}
\end{verbatim}

and you (don't really) have to compile with \info{pdflatex -shell-escape}.

%--------------------------------------------------------------------------------------------------------------------------- 
\subsection{Structure of your \phystricks\ file}
%---------------------------------------------------------------------------------------------------------------------------

Most of your \phystricks files will have the following structure :

\lstinputlisting{phystricksQLXFooBDalHMaT.py}

We will see later the significance of these lines.

%+++++++++++++++++++++++++++++++++++++++++++++++++++++++++++++++++++++++++++++++++++++++++++++++++++++++++++++++++++++++++++
\section{Draw points}
%+++++++++++++++++++++++++++++++++++++++++++++++++++++++++++++++++++++++++++++++++++++++++++++++++++++++++++++++++++++++++++

Here is the code :

\lstinputlisting{phystricksOnePoint.py}


\begin{center}
   \input{Fig_OnePoint.pstricks}
\end{center}



