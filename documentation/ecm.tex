\documentclass{article}
\usepackage{pstricks,pst-eucl,pstricks-add}
\usepackage{pst-plot}
\usepackage{pst-eps}


\pagestyle{empty}

\begin{document}

\begin{TeXtoEPS}
\begin{pspicture}(0,0)(1.0,1.0)
	\pstGeonode[PointSymbol=*,linestyle=solid,linecolor=black](1.0,1.0){B}
	\rput(B){\rput(0.3;180){$q$}}
\end{pspicture}
\end{TeXtoEPS}

\end{document}

latex ecm.tex
dvips ecm.dvi -E -o ecm.eps
pstopdf ecm.eps

Proposé sur le forum pour obtenir le eps:
latex test.tex
dvips test.dvi
ps2eps test.ps 

Autre truc sur le forum:
latex ecm.tex
dvips -E ecm.dvi -o ecm.eps
epstopdf ecm.eps
http://groups.google.fr/group/fr.comp.text.tex/browse_thread/thread/a5c4a67c457c46b8?hl=fr#
