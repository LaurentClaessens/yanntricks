%+++++++++++++++++++++++++++++++++++++++++++++++++++++++++++++++++++++++++++++++++++++++++++++++++++++++++++++++++++++++++++
\section{Installation}
%+++++++++++++++++++++++++++++++++++++++++++++++++++++++++++++++++++++++++++++++++++++++++++++++++++++++++++++++++++++++++++

The module uses \href{http://sagemath.org}{Sage} as backend for the computations. You \emph{need} to have it installed before to try to use phystricks. If you're using Ubuntu, you shouldn't use the package which are much buggy. You have to \href{http://www.sagemath.fr/linux/32bit/index.html}{download} them on the Sage's website.

You also need to have a correct installation of \LaTeX\ and \verb+pstricks+. Your \LaTeX\ file must contain
\begin{verbatim}
\usepackage{pstricks,pst-eucl,pstricks-add}
\end{verbatim}
If you want \verb+eps+ export, you will also need the package 
\begin{verbatim}
\usepackage{graphicx}
\end{verbatim}

The module itself can be downloaded \href{http://student.ulb.ac.be/~lclaesse/}{from here}\footnote{\url{http://student.ulb.ac.be/~lclaesse/phystricks-doc.pdf}} as compressed file containing the source code of this manual.

%+++++++++++++++++++++++++++++++++++++++++++++++++++++++++++++++++++++++++++++++++++++++++++++++++++++++++++++++++++++++++++
\section{Purpose of the module}
%+++++++++++++++++++++++++++++++++++++++++++++++++++++++++++++++++++++++++++++++++++++++++++++++++++++++++++++++++++++++++++

This module is aimed to create pstricks files ready to be included in \LaTeX\ documents. Although it is possible to create simple psfigure environments, it is mainly intended in creating figures in which the picture is included.

Typically, the \LaTeX\ file contains the lines
\begin{verbatim}
My picture is on figure \ref{LabelMyFigure}
\newcommand{\CaptionFigMyFigure}{The caption of my figure}
\input{Fig_MyFigure.pstricks}
\end{verbatim}
and a script using the module phystricks creates the file \verb+Fig_MyFigure.pstricks+.

Thus the compilation is twofold. In a first time, you have to launch your python's script which creates the file \verb+Fig_MyFigure.pstricks+ with the help of the module phystricks. In a second time, you have to compile your \LaTeX\ file.


%+++++++++++++++++++++++++++++++++++++++++++++++++++++++++++++++++++++++++++++++++++++++++++++++++++++++++++++++++++++++++++
\section{Fighting with pdflatex and arxiv}
%+++++++++++++++++++++++++++++++++++++++++++++++++++++++++++++++++++++++++++++++++++++++++++++++++++++++++++++++++++++++++++

One of the main problem with \verb+pstricks+ is that it is not very friendly with \verb+pdflatex+. I \href{http://arxiv.org/abs/0912.2267}{noticed} for example that \href{www.arxiv.org}{arxiv} was unable to process papers with figures created by \verb+phystricks+.

The solution is to run your script with the option \verb+--eps+. This will generate an \verb+eps+ file, and modify the file \verb+.pstricks+ in such a way that it does no more contain \verb+pstricks+ code, but an \verb+\includegraphics+ of the \verb+eps+ picture. Thus you don't need to modify your \LaTeX file, neither you python script. You only have to call the script with the option \verb+--eps+.

If you want to use that functionality, it is necessary to give a name to each of your pictures (in order to name the \verb+eps+ files). Thus, instead of writing
\begin{verbatim}
pspict = pspicture()
\end{verbatim}
you need to write
\begin{verbatim}
pspict = pspicture("Foo")
\end{verbatim}

Then invoking
\begin{verbatim}
you_script --eps
\end{verbatim}
will create the \verb+eps+ file \verb+Picture_Foo-for_eps.eps+ and the file \verb+Fig_Foo.pstricks+ will include that picture by \verb+\includegraphics{Picture_Foo-for_eps.eps}+. Following the \href{http://arxiv.org/help/submit\_tex}{instructions}, it will be sufficient to submit to arxiv.

At this point, your \LaTeX\ file is not yet ready for a compilation by pdf\LaTeX. In order to have a file that can be processed by pdf\LaTeX, you need to invoke
\begin{verbatim}
you_script --pdf
\end{verbatim}
No changes in the script or in the \LaTeX file.


There are, however, still some problems
\begin{enumerate}

	\item
		The conversions create lot of intermediate files.
	\item
		Notice that this documentation cannot be compiled without \verb+pstricks+ because of the table of page \pageref{PgTableauMarques}.
	\item
		Conversion to \verb+pdf+ is not perfect: the bounding box is so strict that if you have a point marked by \verb+*+ right on the corner, the point will be cut. The solution would be to enlarge the bounding box.

\end{enumerate}


